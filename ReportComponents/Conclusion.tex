\chapter{Conclusion}
Overall, making use of Unity ECS, Job System and Burst provides a range of benefits and drawbacks. The results in this project show that Unity ECS with Job System and Burst can perform substantially better than a job optimised standard Unity implementation. In this project, the best Unity ECS implementation consists of 100 000 moving star meshes while the best standard Unity implementation has 10 000. Both implementations had a framerate of \textasciitilde72 FPS. Outside of performance, the data-oriented design of Unity ECS can also result in more modular and reusable code. 

On the other side, using Unity ECS results in a different workflow which is fairly different from the object-oriented design that many are used to. This means that you will have to think and architecture your solutions differently. A fair amount of core engine functionality will have to be written by yourself in the current state of Unity ECS. You might also need to write more boilerplate code to set things up. 

In its current state, I would recommend developers to take a look at it if they want to work with Unity games that require the extra performance. Developers who like writing large portions of the engine by themselves would also benefit from learning Unity ECS. While it is still in preview, starting to learn Unity ECS at this point can provide a good basis for the future. Unity ECS might not be for everyone, but it can still benefit the average developer in some cases. For example, if you are making a simple 2D game you might not need Unity ECS for a PC platform, but it can be useful when working on platforms like mobile devices to conserve battery and maximise hardware usage efficiency. 
