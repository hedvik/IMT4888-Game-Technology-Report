\section{Introduction}
* Unity is a pretty big and popular game engine
* Particularly we see it a lot in the space of indie and mid budget type games
* We dont see it used a lot for AAA development though compared to competitors like Unreal 
* Why does Unity not see that much use for AAA development?
   * One of the reasons for this is a performance/control over performance issue (Needs citations maybe? From a performance point of view this isn't really needed though)
      * Garbage collector can create unwanted stutters in places where you dont want it.
      * Unity has not really had any official support for multithreaded code either
         * It has been possible to use C\# Jobs, but this approach is very limited as interfacing with Unity APIs is only possible on the main thread

* As a means to deal with this problem Unity has recently launched their new Entity Component System (Shortened to ECS) in conjunction with access to Unity's native job system and the new burst compiler. 
   
* The primary goal with ECS is to provide an alternative data-oriented workflow for developers that focuses on performance by default and allowing for a larger degree of control and optimization freedom.
* ECS is pretty new(and in early preview) and it is interesting to see how it compares to regular Unity on a performance side. Given their focus on "performance by default" it is also interesting to see how big the difference in performance is relative to the complexity of actually writing the code.
* The core ECS functionality is planned for release in 2019(according to their latest R/D talk) so this can be seen as a early sneek peek of its capabilities

TODO: 
The report should contain:
* Course Name, University logo(might switch template for that)
* Student Name
* Technology that is being presented

* The Users:
  * Who is using the technology (well, currently just people who are interested since it is not production ready :p)
  * What are they using it for (better performance)
     * (for mobile: lower battery consumption is also a benefit)
  * Why are they using it instead of other approaches (performance, synergy between ECS/JobSystem/Burst)
